% Options for packages loaded elsewhere
\PassOptionsToPackage{unicode}{hyperref}
\PassOptionsToPackage{hyphens}{url}
%
\documentclass[
]{article}
\usepackage{lmodern}
\usepackage{amssymb,amsmath}
\usepackage{ifxetex,ifluatex}
\ifnum 0\ifxetex 1\fi\ifluatex 1\fi=0 % if pdftex
  \usepackage[T1]{fontenc}
  \usepackage[utf8]{inputenc}
  \usepackage{textcomp} % provide euro and other symbols
\else % if luatex or xetex
  \usepackage{unicode-math}
  \defaultfontfeatures{Scale=MatchLowercase}
  \defaultfontfeatures[\rmfamily]{Ligatures=TeX,Scale=1}
\fi
% Use upquote if available, for straight quotes in verbatim environments
\IfFileExists{upquote.sty}{\usepackage{upquote}}{}
\IfFileExists{microtype.sty}{% use microtype if available
  \usepackage[]{microtype}
  \UseMicrotypeSet[protrusion]{basicmath} % disable protrusion for tt fonts
}{}
\makeatletter
\@ifundefined{KOMAClassName}{% if non-KOMA class
  \IfFileExists{parskip.sty}{%
    \usepackage{parskip}
  }{% else
    \setlength{\parindent}{0pt}
    \setlength{\parskip}{6pt plus 2pt minus 1pt}}
}{% if KOMA class
  \KOMAoptions{parskip=half}}
\makeatother
\usepackage{xcolor}
\IfFileExists{xurl.sty}{\usepackage{xurl}}{} % add URL line breaks if available
\IfFileExists{bookmark.sty}{\usepackage{bookmark}}{\usepackage{hyperref}}
\hypersetup{
  pdftitle={Untitled},
  pdfauthor={Pablo Hidalgo García},
  hidelinks,
  pdfcreator={LaTeX via pandoc}}
\urlstyle{same} % disable monospaced font for URLs
\usepackage[margin=1in]{geometry}
\usepackage{graphicx,grffile}
\makeatletter
\def\maxwidth{\ifdim\Gin@nat@width>\linewidth\linewidth\else\Gin@nat@width\fi}
\def\maxheight{\ifdim\Gin@nat@height>\textheight\textheight\else\Gin@nat@height\fi}
\makeatother
% Scale images if necessary, so that they will not overflow the page
% margins by default, and it is still possible to overwrite the defaults
% using explicit options in \includegraphics[width, height, ...]{}
\setkeys{Gin}{width=\maxwidth,height=\maxheight,keepaspectratio}
% Set default figure placement to htbp
\makeatletter
\def\fps@figure{htbp}
\makeatother
\setlength{\emergencystretch}{3em} % prevent overfull lines
\providecommand{\tightlist}{%
  \setlength{\itemsep}{0pt}\setlength{\parskip}{0pt}}
\setcounter{secnumdepth}{5}

\title{Untitled}
\author{Pablo Hidalgo García}
\date{29/7/2020}

\begin{document}
\maketitle

\hypertarget{introducciuxf3n}{%
\section{Introducción}\label{introducciuxf3n}}

En la compleja sociedad moderna aparecen retos que es necesario abordar
y solucionar de la mejor forma posible para que la vida sea fácil y
llevadera. Además, existe una conciencia creciente acerca del impacto
medioambiental de nuestras actividades por lo que las soluciones a estos
retos deben tenerlo en consideración. Uno de esos muchos retos es la
gestión de los residuos. La sociedad de consumo moderna implica una alta
generación de residuos que es necesario procesar para reducir la huella
medioambiental así como evitar la aparición de enfermedades o las
incomodidades propias de la convivencia con los residuos. En el año 2017
se recogieron en España más de 22.000 toneladas de residuos (alrededor
de 460 kilogramos por habitante) y que da cuenta de la magnitud y la
dificultad en la gestión. En áreas rurales o insulares, esta gestión es
complicada en el sentido de que los trayectos pueden ser amplios y la
frecuencia de recogida no puede ser diaria adaptándose la recogida al
comportamiento de la generación de los residuos. Por ello, es
fundamental que la recogida de estos residuos se haga de la forma más
eficiente posible. Se pueden distinguir dos formas en la que la recogida
de los residuos se puede mejorar. La primera de ellas es realizar
inversiones que consigan adaptar los recursos e infraestructuras
disponibles (camiones de recogida, puntos de recogida, plantas de
procesamiento, etcétera) a los patrones de generación de residuos; la
segunda es optimizar la recogida contando con los recursos ya
existentes.

Es en este segundo enfoque en el que se incide en este trabajo, en
particular en optimizar las rutas que recorren los camiones con el
objetivo de recoger la máxima cantidad de residuo conforme a las
restricciones de recursos. Las restricciones son:

\begin{itemize}
\tightlist
\item
  plantas de procesamiento,
\item
  puntos de recogida,
\item
  camiones,
\item
  horas de trabajo.
\end{itemize}

De estas restricciones la más fuerte es el número máximo de horas de
trabajo que se consideran de 6.5 horas dejando un margen para cualquier
eventualidad. Esta restricción es obvia para que los trabajadores
descansen adecuadamente y, además, por la naturaleza de la recogida de
residuos, se fomenta la recogida durante los periodos nocturnos para
molestar lo menos posible a la población.

Este trabajo se desarrolla en este contexto tomando como escenario de
estudio la isla de la Palma (Islas Canarias) aunque su aplicación se
puede extender y adaptar a cualquier otra área geográfica. La
información disponible son: patrones de generación de residuos,
emplazamiento de los contenedores y sus características, detalles de las
rutas, tiempo y coste y las restricciones de los recursos disponibles.
Existen tres puntos de origen y destino de las rutas situadas en tres
municipios de la Palma: Breña Alta, Mazo y Los Llanos. Cada vehículo
debe llevar a cabo una ruta diaria y, cada ruta se define como un
secuencia de putnos de recogida que deben ser visitados por cada
vehículo. Se considera que cada vez que un camión visita un punto de
recogida, éste recoge todo el residuo acumulado.

Las principales contribuciones de este trabajo son:

\begin{enumerate}
\def\labelenumi{\arabic{enumi}.}
\tightlist
\item
  Propuesta de un algoritmo metaheurístico híbrido entre una búsqudda
  tabú y una búsqueda por vecindarios variables.
\item
  Comparativa del algoritmo propuesto con la ruta real y con EXPÓSITO.
\item
  Estudio del algoritmo bajo distintos escenarios.
\end{enumerate}

Este trabajo está organizado de la siguiente forma. La sección 2 hace un
repaso de la literatura existente

En el problema se han considerado al recogida independiente de dos tipos
de residuos: residuos de papel y cartón y residuos plásticos
identificados, habitualmente, con los colores marrones y amarillos,
respectivamente. Para cada tipo de residuo se considera un camión con un
punto de origen y un punto de destino (el almacén y la planta de
tratamiento de residuos) y una serie de puntos de recogida candidatos.
Cada punto de recogida tiene su propia tasa de llenado que determina el
nivel de residuo hasta el punto en el que esté saturado. La restricción
principal que hay que abordar es la del tiempo máximo de trabajo que los
operarios considerando un tiempo de 6.5 horas.

Este trabajo toma como punto de partida el ya iniciado en EXPÓSITO. En
ese trabajo se aplica modelos metaheurísticos GRASP (greedy randomized
adaptative search procedure) para abordar el problema. En este trabajo
se se desarrolla un algoritmo híbrido entre la búsqueda tabú y la
búsqueda por vecindarios. Además, en EXPÓSITO, los resultados obtenidos
se comparan con las rutas existentes en la compañía de recogida de
residuos en una semana concreta. En este trabajo, el estudio se extiende
para evaluar el comportamiento del algoritmo para distintos escenarios
(condiciones iniciales distintas así como distintos horizontes
temporales.

El objetivo de este trabajo es el de aplicar algoritmos metaheurísticos
en la gestión de las rutas de los residuos como alternativa a lo
estudiado en (Expósito).

En este trabajo se ha utilizado un algoritmo híbrido entre la búsqueda
tabú y la búsqueda por vecindarios variables (Variable Neighborhood
search o VNS).

Durante el año 2017 se recogieron en España más de 22.000 toneladas de
residuos. La eficiencia en su recogida hace que

En el conjunto de la sociedad se presentan retos que es necesario
abordar. Por ejemplo, la gestión eficiente del tráfico o la gestión de
los residuos.

Vivimos en una sociedad cada vez más preocupada por el impacto

\textbf{Esquema de la introducción}

\begin{enumerate}
\def\labelenumi{\arabic{enumi}.}
\tightlist
\item
  Problema de la gestión de residuos en general.
\item
  Características de la gestión en zonas rurales o remotas.
\item
  Descripción del problema del trabajo.
\item
  Pinceladas de la solución.
\item
  Relación con EXPÓSITO.
\item
  Novedades de este trabajo.
\end{enumerate}

\hypertarget{revisiuxf3n-bibliogruxe1fica}{%
\section{Revisión bibliográfica}\label{revisiuxf3n-bibliogruxe1fica}}

El problema de la recogida de residuos se engloba dentro de los
problemas conocidos como Problema de enrutamiento de vehículos (Vehicle
Routing Problem o VRT) enunciado por primera vez en (Dantzig and Ramser
1959) y que puede cosiderarse como una generalización del problema del
viajante (Traveling-Salesman Problem o TSP) ({\textbf{???}}). En el
problema del viajante (TSP en lo que sigue) busca la ruta óptima entre
\(n\) puntos pasando una única vez por cada uno de ellos. El problema de
enrutamiento de vehículos (VRP en lo que sigue) trata de encontrar la
mejor ruta para una flota de vehículos (y no solo uno, como en el TSP)
para satisfacer la demanda de un conjunto de clientes conforme a un
criterio de optimización. Estos dos problemas son fundamentales en el
desarrollo de la inteligencia artificial por la cantidad de aplicaciones
prácticas. De hecho, se suelen nombrar algunas variaciones de este
problema:

\begin{enumerate}
\def\labelenumi{(\roman{enumi})}
\tightlist
\item
  Capacitated VRP: está disponible una flota homogénea de vehículos
  donde la restricción es la capacidad de cada vehículo,
\item
  VRP con ventana de tiempos: los clientes tienen que ser servidos en un
  intervalo de tiempo específico.
\end{enumerate}

En los problemas VRP tradicionales, el objetivo se trata desde un punto
de vista del impacto económico (habitualmente, se define algún tipo de
coste). Sin embargo, han surgido una familia de problemas denominados
\emph{Green Vehicle Routing Problems} (GVRP) caracterizados por tener en
cuenta la armonización entre el impacto económico y medioambiental.
Dentro de esta familia de problemas, se pueden ver tres grandes
categorías (Lin et al. 2014):

\begin{itemize}
\tightlist
\item
  G-VRP (Green-VRP): el objetivo es optimizar el consumo de la energía
  necesaria para el transporte,
\item
  Pollution Routing Problem (PRP): el objetivo es encontrar la mejor
  planificación de rutas desde el punto de vista de la contaminación, en
  particular, reduciendo las emisiones de carbono.
\item
  VRP en Reverse Logistics (VRPRL): estos problemas están relacionados
  con aspectos de la logística inversa (traslado de materiales apra su
  reciclado, reutilización o destrucción).
\end{itemize}

El problema de la recogida de residuos se trata en (Expósito-Márquez et
al. 2019) donde se define formalmente el problema y se desarrolla una
solución mediante el algoritmo Feedy Randomized Adaptative Search
Procedure (GRASP).

\begin{itemize}
\tightlist
\item
  Problema:

  \begin{itemize}
  \tightlist
  \item
    TSP -\textgreater{} variantes
  \end{itemize}
\item
  Métodos metaheurísticos. En particular búsqueda Tabú y VNS
\end{itemize}

\hypertarget{descripciuxf3n-del-problema}{%
\section{Descripción del problema}\label{descripciuxf3n-del-problema}}

El problema se puede describir formalmente como un grafo completo
dirigido \(\mathcal{G} = (\Theta, A)\) donde
\(\Theta=\{\theta_1,\ldots, \theta_n\}\) se corresponde con el conjunto
de los \(n\) emplazamientos y
\(A=\{(\theta_i,\theta_j):\theta_i,\theta_j\in\Theta, i\neq j\}\), las
aristas del grafo. Además, el conjunto \(\Theta = P \cup E\) de forma
que \(P\cap E = \emptyset\). Conocemos el tiempo de viaje entre dos
emplazamientos \(\theta_i, \theta_j\) al que denominaremos \(d_{ij}>0\)
y que, debido a la orografía del terreno y la infraestructura de
carreteras, en general, \(d_{ij}\neq d_{ji}\).

El conjunto de los puntos de recogida se recogen por una flota de
vehículos \(\mathcal{V} = \{v_1,\ldots, v_k\}\). El origen del vehículo
\(v\in\mathcal{V}\) se denomina como \(o(v)\in E\) y \(t(v)\in E\) al
destino final del vehículo.

Se define un horizonte de predicción \(\mathcal{H}=\{1,2,\ldots, h\}\).

\[
F_i(d) = \min\{1, f_i(d)\}
\]

Aunque la función de residuos acumulados \(f_i(d)\) puede tomar
cualquier forma, en este trabajo se ha considerado una función lineal
\(f_i(d) = b_i + q_i\cdot d\)

\hypertarget{formulaciuxf3n-matemuxe1tica}{%
\section{Formulación matemática}\label{formulaciuxf3n-matemuxe1tica}}

En esta sección expresaremos el problema de optimización que queremos
resolver. Éste se puede formular como un problema de programación entera
mixta (Mixed-Integer Programming o MIP). Necesitamos las siguientes
variables:

\begin{itemize}
\tightlist
\item
  \(X_{ijh}^v\): variable que toma valor \(1\) si el vehículo
  \(v\in\mathcal{V}\) va desde el punto \(\theta_i\) hasta el
  \(theta_j\) en el día \(h\in \mathcal{H}\) y 0 en caso contrario,
  \(\forall (\theta_i,\theta_j) \in A\)
\item
  \(Y_{ih}^v\): toma valor \(1\) si se visita el punto
  \(\theta_i\in\Theta\) por el vehículo \(v\in\mathcal{V}\) en el día
  \(h\in\mathcal{H}\) y 0 en caso contrario.
\item
  \(\mathcal{T}_{ih}\in\mathbb{R}\): tiempo de recogida del punto
  \(\theta_i\in\Theta\). Nótese que esta variable no depende de ningún
  vehículo ya que se considera que un punto de recogida solo se puede
  visitar por un único vehículo en un mismo día.
\end{itemize}

Así, las restricciones del problema son las siguientes:

\[
\sum_{j \in P} X_{o(v)jh}^v= 1 
\]

\[
\sum_{i \in P} X_{it(v)h}^v= 1 
\]

\[
\sum_{j\in\Theta}X_{jkh}^v = \sum_{j\in\Theta}X_{kjh}^v, k\in P,   
\]

\[
\sum_{j\in\Theta} X_{ijh}^v = Y_{ih}^v
\] \[
\sum_{v\in \mathcal{V}} Y_{ih}^v \leq 1
\]

\[
T_{jh} \geq T_{ih} + s_i + t_{ij} - M\cdot \big(1-\sum_{v\in \mathcal{V}} X_{ijh}^v\big)
\]

\[
T_{ih} + s_i + t_{it(v)} \leq W_{vh}
\]

\[
X_{ijh}^v \in \{0, 1\}
\]

\[
Y_{ih}^v \in \{0, 1\}
\] \[
T_{ih} \geq 0 
\]

El objetivo es

\[
\max \sum_{v \in V}\sum_{i\in P} \sum_{h\in H} F_i(\eta_{ih})\cdot Y_{ih}^v
\]

\hypertarget{solution-approach}{%
\section{Solution approach}\label{solution-approach}}

En esta sección se describe la solución propuesta. La formulación
matemática anterior es intratable computacionalmente cuando la
dimensionalidad de los escenarios es alta (Expósito-Márquez et al.
2019), como es el caso del escenario de recogida de residuos en la isla
de la Palma que utilizamos aquí.

La solución propuesta busca diseñar las rutas que deben seguir cada
vehículo cada uno de los días contenidos en el horizonte temporal. Para
ello se aplica en la marco general de la búsqueda en vecindarios
variables o VNS (Mladenović and Hansen 1997) es un algoritmo
metaheurístico para resolver problemas de optimización combinatorios y
globales cuya principal idea es el cambio sistemático del vecindario de
búsqueda tanto en un fase de búsqueda local como de una fase de
perturbación que permita escapar de óptimos locales. Las distintas
extensiones de este algoritmo se pueden consultar en (Hansen et al.
2010). El algoritmo VNS está basado en tres hechos fundamentales: (1) un
mínimo local con respecto a una estructura de vecindario no lo es
necesariamente respecto de otra, (2) un mínimo global es un mínimo local
con respecto a todas las estructuras de vecindario posibles y (3) para
muchos problemas, el mínimo local con respecto a uno o varios
vecindarios están relativamente cercanos unos de otros.

Aunque hay diferentes variantes del algoritmo VNS, una de las
aproximaciones que más éxito han tenido es es la denominada General VNS
(Hansen et al. 2010). Para definir este algoritmo comenzamos definiendo
la función \texttt{NeighborhoodChange} que permite comparar los valores
de dos soluciones de forma que, en caso de que la nueva solución sea
mejor que la anterior, se actualiza la solución y se vuelve a usar el
vecindario original \(\mathcal{N}_1\). En caso contrario, se utiliza el
siguiente vecindario.

\textbf{Inicialización.} Seleccionar el conjunto de las estructuras de
vecindarios \(\mathcal{N}_k\) para \(k=1,\ldots, k_{max}\) que se usará
en la búsqueda; encontrar una solución inicial \(x\) y su valor de la
función objetivo \(f(x)\); asignar
\(x_{opt}\leftarrow x, f_{opt}\leftarrow f(x)\); elegir un critero de
parada.

\textbf{Repetir} lo siguiente hasta que se satisfaga el criterio de
parada:

\begin{enumerate}
\def\labelenumi{(\arabic{enumi})}
\tightlist
\item
  Asignar \(k\leftarrow 1\);
\item
  Repetir hasta que \(k=k_{max}\):

  \begin{enumerate}
  \def\labelenumii{(\alph{enumii})}
  \tightlist
  \item
    \textbf{Agitación}. Generar un punto \(x'\) aleatoriamente para el
    vecindario \(\mathcal{N}_k(x)\);
  \item
    \(l\leftarrow 1\);
  \item
    Repetir hasta que \(l=l_{max}\):

    \begin{enumerate}
    \def\labelenumiii{(\roman{enumiii})}
    \tightlist
    \item
      \textbf{Búsqueda local.} Encontrar el mejor vecino en
      \(\mathcal{N}_l(x')\) y denotarlo como \(x''\);
    \item
      \textbf{Cambio de vecindario.} Si \(f(x'') > f_{opt}\) hacer
      \(f_{opt}\leftarrow f(x'')\), \(x_{opt}\leftarrow x''\) y
      \(l\leftarrow l_{max}\); en otro caso \(l\leftarrow l + 1\);
    \end{enumerate}
  \item
    \textbf{Mover o no.} Si \(f(x'') < f_{opt}\), asignar
    \(k \leftarrow k + 1\); en caso contrario, \(k\leftarrow 1\).
  \end{enumerate}
\end{enumerate}

Este algoritmo GVNS tiene varias extensiones, entre ellas la hibridación
con otros algoritmos. Uno de los algoritmos metaheurísticos más
utilizado es la búsqueda tabú. Esta algoritmo mantiene una
\emph{memoria} acerca de de la búsqueda ya realizada para prohibir
(tabú) algunas soluciones recientes. Aunque se puede hibridar con GVNS
de distintas formas, en este trabajo utilizaremos una hibridación de
forma que el conjunto de soluciones vecinas esté limitado por la lista
tabú. Así, el algoritmo anterior quedaría como

\textbf{Inicialización.} Seleccionar el conjunto de las estructuras de
vecindarios \(\mathcal{N}_k\) para \(k=1,\ldots, k_{max}\) que se usará
en la búsqueda; encontrar una solución inicial \(x\) y su valor de la
función objetivo \(f(x)\); asignar
\(x_{opt}\leftarrow x, f_{opt}\leftarrow f(x)\); elegir un critero de
parada; definir un conjunto de soluciones tabú \(\mathcal{T}\).

\textbf{Repetir} lo siguiente hasta que se satisfaga el criterio de
parada:

\begin{enumerate}
\def\labelenumi{(\arabic{enumi})}
\tightlist
\item
  Asignar \(k\leftarrow 1\);
\item
  Repetir hasta que \(k=k_{max}\):

  \begin{enumerate}
  \def\labelenumii{(\alph{enumii})}
  \tightlist
  \item
    \textbf{Agitación}. Generar un punto \(x'\) aleatoriamente para el
    vecindario \(\mathcal{N}_k(x, \mathcal{T})\);
  \item
    \(l\leftarrow 1\);
  \item
    Repetir hasta que \(l=l_{max}\):

    \begin{enumerate}
    \def\labelenumiii{(\roman{enumiii})}
    \tightlist
    \item
      \textbf{Búsqueda local.} Encontrar el mejor vecino en
      \(\mathcal{N}_l(x')\) y denotarlo como \(x''\);
    \item
      \textbf{Cambio de vecindario.} Si \(f(x'') > f_{opt}\) hacer
      \(f_{opt}\leftarrow f(x'')\), \(x_{opt}\leftarrow x''\),
      \(l\leftarrow l_{max}\) y actualizar el conjunto tabú
      \(\mathcal{T}\); en otro caso \(l\leftarrow l + 1\);
    \end{enumerate}
  \item
    \textbf{Mover o no.} Si \(f(x'') < f_{opt}\), asignar
    \(k \leftarrow k + 1\); en caso contrario, \(k\leftarrow 1\).
  \end{enumerate}
\end{enumerate}

\hypertarget{caracteruxedsticas-del-problema}{%
\section{Características del
problema}\label{caracteruxedsticas-del-problema}}

EL algoritmo definido en la sección anterior se puede aplicar a
cualquier problema combinatorio. Para la aplicación del problema de
optimización de recogida de residuos enunciado es necesario definir la
estructura de los vecindarios y cómo se va a ir modificando la lista
tabú.

Se consideran tres estructuras de vecindarios.

\begin{itemize}
\item
  Vecindario \(\mathcal{N}_1\): dada una ruta, se añade un nuevo punto
  de recogida teniendo en cuenta que un punto de recogida no debe
  aparecer en dos rutas del mismo día.
\item
  Vecindario \(\mathcal{N}_2\): se intercambian dos puntos de recogida
  entre dos días distintos.
\item
  Vecindario \(\mathcal{N}_3\): se intercambia un punto de recogida que
  aparece en una de las rutas por uno que no haya sido recogido teniendo
  en cuenta, de nuevo, que un punto de recogida no puede ser recogido
  dos veces en el mismo día.
\end{itemize}

Cabe señalar que una de las principales restricciones del problema es el
máximo tiempo que puede durar una ruta. Por tanto, en la búsqueda local
en el vecindario de una solución, en el caso de que se encuentre algún
vecino que, aunque tenga un valor de la función objetivo igual que la
mejor solución, si ésta supone una disminución en el tiempo total de las
rutas, se seleccionará esta solución.

Cada vez que se añade un punto de recogida a la solución, éste se
introduce en la lista tabú \(\mathcal{T}\). Además, es necesario definir
un parámetro \(\rho\) que especifique durante cuántas iteraciones un
punto de recogida no se considerará desde que fue introducido en la
lista \(\mathcal{T}\).

En realidad, el problema se puede dividir en dos partes: (1) asignar un
punto de recogida a una ruta y (2) elegir el orden adecuado en el que se
deben visitar los puntos de recogida en esa ruta. Para esta segunda
parte, se realiza una estrategia de mejora de la ruta mediante el
heurístico Lin-Kernighan (Helsgaun 2000) el cual se considera uno de las
mejores heurísticas para aplicar al problema TSP. Este heurístico se
implementa dentro de las soluciones obtenidas en las distintas
estructuras de vecindario.

\begin{itemize}
\tightlist
\item
  \(X = \mathcal{N}_k(x)\)
\item
  Repetir hasta encontrar una solución que verifique las restricciones:

  \begin{itemize}
  \tightlist
  \item
    Asignar a \(x'\) la solución de \(X\) con un mayor valor de la
    función objetivo; en caso de que haya varias soluciones posibles
    \(x'\), se escoge aquella que tenga un menor tiempo total en las
    rutas.
  \item
    Aplicar la heurística Lin-Kernighan a \(x'\).
  \item
    Si \(x'\) no cumple las restricciones,
    \(\mathcal{N}_k(x)\leftarrow \mathcal{N}_k(x) \setminus \{x'\}\)
  \end{itemize}
\end{itemize}

\textbf{Algoritmo shake}

Comenzamos definiendo los vecindarios.

Una ruta \(r_h^v = (o(v), t(v))\)

\hypertarget{estructuras-de-vecindarios}{%
\section{Estructuras de vecindarios}\label{estructuras-de-vecindarios}}

Se definen tres estructuras de vecindarios:

\begin{itemize}
\tightlist
\item
  \(\mathcal{N}_1\): añadir nuevos puntos de recogida.
\end{itemize}

Los vecinos de una solución se obtienen añadiendo un nuevo punto de
recogida a las rutas.

\begin{enumerate}
\def\labelenumi{\arabic{enumi}.}
\tightlist
\item
  while improvement do:

  \begin{enumerate}
  \def\labelenumii{\arabic{enumii}.}
  \tightlist
  \item
    Improvement \textless- False
  \item
    For \(\theta \in candidates\) do:

    \begin{enumerate}
    \def\labelenumiii{\arabic{enumiii}.}
    \tightlist
    \item
      For \(v\in V\) do:
    \item
      For \(h\in \mathcal{H}\) do:
    \end{enumerate}
  \end{enumerate}
\end{enumerate}

\begin{itemize}
\tightlist
\item
  \(\mathcal{N}_2\): intercambiar puntos de recogida.
\end{itemize}

Cada vecino se define como el intercambio entre rutas de puntos de
recogida existentes. Es decir, dados dos camiones
\(v, v'\in \mathcal{V}\) (que puede darse \(v = v'\)) y dos días
\(d, d' \in \mathcal{H}\) de forma que \(d\neq d'\), se escogen dos
puntos de recogida \(\theta\in R_{d}^v\) y \(\theta\in R_{d'}^{v'}\) y
se intercambian entre ellos.

\begin{itemize}
\tightlist
\item
  \(\mathcal{N}_2\): cambiar puntos de recogida actuales por otros no
  considerados.
\end{itemize}

Los vecinos se obtienen al intercambiar un punto de recogida visitado en
la solución actual en un día por otro que no haya sido visitado en ese
día.

\begin{enumerate}
\def\labelenumi{\arabic{enumi}.}
\tightlist
\item
  Inicialización:

  \begin{itemize}
  \tightlist
  \item
    \(P_h=\,\) conjunto de puntos
  \item
    \(\hat{\Theta}_h\): conjunto de puntos de recogida que no se visitan
    en el día \(h\).
  \end{itemize}
\item
  Repetir hasta mejorar:
\end{enumerate}

\hypertarget{vecindario-1-auxf1adir-punto-de-recogida}{%
\subsection{Vecindario 1: añadir punto de
recogida}\label{vecindario-1-auxf1adir-punto-de-recogida}}

\hypertarget{vecindario-2-intercambiar-puntos-de-recogida-entre-duxedas}{%
\subsection{Vecindario 2: intercambiar puntos de recogida entre
días}\label{vecindario-2-intercambiar-puntos-de-recogida-entre-duxedas}}

\hypertarget{vecindario-3-intercambiar-punto-de-recogida-con-uno-no-visitado}{%
\subsection{Vecindario 3: intercambiar punto de recogida con uno no
visitado}\label{vecindario-3-intercambiar-punto-de-recogida-con-uno-no-visitado}}

\hypertarget{vecindarios}{%
\section{Vecindarios}\label{vecindarios}}

\hypertarget{vecindario-1-auxf1adir}{%
\subsection{Vecindario 1: añadir}\label{vecindario-1-auxf1adir}}

Sea \(h\in\mathcal{H}\) un día del horizonte y \(r_h\) la ruta de ese
día tal que \(r_i=(o(\nu), p_1, p_2, \ldots, p_n, t(\nu))\).
Representamos todas las rutas como \(R=\{r_1, r_2,\ldots, r_{|H|}\}\).

Este vecindario busca añadir un nuevo punto \(p\in P(\nu)\) tal que el
incremento en la función objetivo sea máximo.

\hypertarget{vecindario-2-intercambiar-entre-duxedas}{%
\subsection{Vecindario 2: intercambiar entre
días}\label{vecindario-2-intercambiar-entre-duxedas}}

\hypertarget{vecindario-3-cambiar-puntos-de-recogida}{%
\subsection{Vecindario 3: cambiar puntos de
recogida}\label{vecindario-3-cambiar-puntos-de-recogida}}

\hypertarget{resultados-computacionales}{%
\section{Resultados computacionales}\label{resultados-computacionales}}

En esta sección se describen los resultados computacionales obtenidos en
el escenario de análisis. Los objetivos de este análisis son:

\begin{enumerate}
\def\labelenumi{\arabic{enumi}.}
\tightlist
\item
  Evaluar el comportamiento de la solución propuesta.
\item
  Comparar los resultados de la solución propuesta con la situación real
  actual y la solución propuesta en (Expósito-Márquez et al. 2019).
\item
  Evaluar el comportamiento de la solución en bajo diversos escenarios.
\end{enumerate}

Los resultados computacionales que aparecen en esta sección se han
realizado en un ordenador equipado con un procesador Intel Core i7-8700
3.20GHz y 16 GB de memoria RAM. La implementación de la solución se ha
realizado utilizando el lenguaje de programación Python 3.8 recogido en
el repositorio \url{https://github.com/papabloblo/tfm}. Se ha
considerado un tiempo de ejecución de 4 horas, un tiempo asumible para
la que la empresa de recogida pueda obbtener la planificación de las
rutas y llevarlas a cabo. Nótese que el horizonte temporal es de 5 días
así que, podría considerarse por parte de la empresa ejecutar el
algoritmo diariamiente para ir recalculando las rutas si eso fuese
posible y necesario utilizando como solución de inicial para el
algoritmo las rutas planificadas.

En la sección 6.1 se describen los datos que componen el escenario bajo
el que se realizan las pruebas computacionales.

\hypertarget{resumen-de-los-datos}{%
\subsection{Resumen de los datos}\label{resumen-de-los-datos}}

El escenario sobre el que se aplica la solución propuesta se realiza
sobre la isla de La Palma (Islas Canarias, España). Esta isla tiene una
extensión de 708.32 \(km^2\) con una población de 82,671 habitantes.
{[}fuente:
\url{https://www.ine.es/jaxiT3/Datos.htm?t=2910\#!tabs-tabla}{]} La isla
tiene como punto más elevado el Roque de los Muchachos (2426 metros).
Está compuesta por 14 municipios. Su particular orografía y las
conexiones entre los municipios, hace que las distancias y tiempos de
viaje no sea simétricos. Además, se trata de una isla con un fuerte
componente turístico son una media de 140,000 turistas anuales
{[}¿CITA?{]}.

Los datos utilizados para el estudio provienen de un caso real de
estudio, descrito en detalle en (Expósito-Márquez et al. 2019). Se
contemplan dos escenarios de recogida de residuos y que constituyen
problemas completamente independientes: residuos de papel y cartón
(asociado al color azul) y residuos plásticos (asociado al color
amarillo). Ambos escenarios comparten 338 puntos de recogida de los que
se puede ver su situación en la figura X.

Los tiempos de viaje entre puntos de recogida se han obtenido a través
de Google Distance Matrix API como una matriz \(T\) de tamaño
\(N\times N\) siendo \(N\) el número de puntos de recogida. Esta matriz
tiene como características: (1) \(t_{ii} = 0\), (2) \(t_{ij} \geq 0\),
\(\forall i\neq j\) y (3), en general, \(t_{ij}\neq t_{ji}\). En la
matriz de distancias, la media de tiempo entre un punto de recogida y el
punto de recogida más cercano es de 109.61 con una desviación típica de
111.47.

Un dato importante en el escenario es la forma en la que los puntos de
recogida se van llenando de residuos. El modelo no impone de qué forma
se debe producir este llenado. Se hace una simplificación de forma que
se supone que el llenado se hace de forma lineal conforme a una tasa
propia de cada punto de recogida. La tasa media de de llenado es de
0.134 con una desviación típica de 0.089 para la recogida de plásticos y
0.188 con una desviación típica de 0.108, para la recogida de cartones.

En ambos escenarios se considera que está disponible, diariamente, dos
vehículos de recogida (uno para cada tipo de resiudo) sobre el que, a
efectos prácticos, no se considera un límite de capacidad de residuos
recogidos.

Para el primer escenario y poder realizar la comparativa con
(Expósito-Márquez et al. 2019), se considera datos obtenidos en la
semana del 2 al 6 de octubre para obtener el nivel de llenado en el día
de partida de la optimización. El tiempo máximo en el que puede durar un
ruta se correpsonde con 6.5 horas.

Los algoritmos descritos anteriormente se aplican en un contexto de
recogida de residuos de la isla de la Palma. Se consideran un total de
XX puntos de recogida entre los que se tiene un tiempo medio de viaje de
XX. Además, tal y como aparece en (Expósito-Márquez et al. 2019), se ha
considerado añadir un tiempo de recogida de cada punto de 120 segundos.
Tiempo máximo de ruta: 6.5 horas.

La orografía hace que los tiempos entre puntos de recogida no sea
simétirca.

\begin{itemize}
\tightlist
\item
  Mapa de los puntos de recogida. {[}¡ESTO HAY QUE PEDIRLO!{]}
\end{itemize}

Para comprobar el comportamiento del algoritmo en otras situaciones a la
descrita anteriormente, se llevan a cabo ejecuciones en la que el
llenado inicial de los puntos de recogida se obtiene aleatoriamente.
Esto permite comprobar con generalidad el comportamiento del algoritmo.
Como la recogida total de residuo se ve afectada por la cantidad de
residuo inicial en cada punto de recogida, para la comparativa se ha
utilizado el indicador en función del.

Sabiendo que el llenado de los puntos de recogida se produce con la
fórmula

\[
r_i = a_i + b_i\cdot d
\] Si no consideramos el desbordamiento de los contenedores, el residuo
acumulado de los puntos de recogida, considerando un horizon de \(h\)
días sería

\[
r_i = a_i + b_i\cdot h
\]

Así, podemos tener el siguiente indicador. Sea \(f\) la cantidad de
residuo recogido por una solución,

\[
indicador = \frac{f}{\sum_{i\in\Theta} r_i}
\] Este indicador nos permitirá obtener el procentaje del residuo
recogido del total de residuo generado durante el horizonte de
planificación y poder comparar los distintos escenarios.

Métrica para calcular los puntos de recogida saturados

Dada una ruta \(r\), será interesante tener en cuenta la satisfacción
del servicio que puedan percibir los usuarios. Un indicador clave es que
los puntos de recogida no estén desbordados lo que conlleva la posible
acumulación de residuos fuera de él o que el usuario tenga que
desplazarse hasta otro punto de recogida cercano que no es esté
desbordado. De aquí podemos obtener dos indicadores:

\begin{itemize}
\tightlist
\item
  Número de puntos de recogida que se desbordan en algún momento durante
  el horizonte de planificación,
\item
  Cantidad de residuo ``desbordado''. (Diferencia entre el residuo
  téorico y el resiudo recogido en el caso de que se produzca un
  desbordamiento)
\end{itemize}

\hypertarget{conclusiones-y-trabajo-futuro}{%
\section*{Conclusiones y trabajo
futuro}\label{conclusiones-y-trabajo-futuro}}
\addcontentsline{toc}{section}{Conclusiones y trabajo futuro}

\hypertarget{refs}{}
\leavevmode\hypertarget{ref-dantzig_truck_1959}{}%
Dantzig, G. B., and J. H. Ramser. 1959. ``The Truck Dispatching
Problem.'' \emph{Management Science} 6 (1): 80--91.
\url{https://doi.org/10.1287/mnsc.6.1.80}.

\leavevmode\hypertarget{ref-exposito-marquez_greedy_2019}{}%
Expósito-Márquez, Airam, Christopher Expósito-Izquierdo, Julio
Brito-Santana, and J. Andrés Moreno-Pérez. 2019. ``Greedy Randomized
Adaptive Search Procedure to Design Waste Collection Routes in La
Palma.'' \emph{Computers \& Industrial Engineering} 137 (November):
106047. \url{https://doi.org/10.1016/j.cie.2019.106047}.

\leavevmode\hypertarget{ref-gendreau_variable_2010}{}%
Hansen, Pierre, Nenad Mladenović, Jack Brimberg, and José A. Moreno
Pérez. 2010. ``Variable Neighborhood Search.'' In \emph{Handbook of
Metaheuristics}, edited by Michel Gendreau and Jean-Yves Potvin,
146:61--86. Boston, MA: Springer US.
\url{https://doi.org/10.1007/978-1-4419-1665-5_3}.

\leavevmode\hypertarget{ref-helsgaun_effective_2000}{}%
Helsgaun, Keld. 2000. ``An Effective Implementation of the
Lin--Kernighan Traveling Salesman Heuristic.'' \emph{European Journal of
Operational Research} 126 (1): 106--30.
\url{https://doi.org/10.1016/S0377-2217(99)00284-2}.

\leavevmode\hypertarget{ref-lin_survey_2014}{}%
Lin, Canhong, K. L. Choy, G. T. S. Ho, S. H. Chung, and H. Y. Lam. 2014.
``Survey of Green Vehicle Routing Problem: Past and Future Trends.''
\emph{Expert Systems with Applications} 41 (4): 1118--38.
\url{https://doi.org/10.1016/j.eswa.2013.07.107}.

\leavevmode\hypertarget{ref-mladenovic_variable_1997}{}%
Mladenović, N., and P. Hansen. 1997. ``Variable Neighborhood Search.''
\emph{Computers \& Operations Research} 24 (11): 1097--1100.
\url{https://doi.org/10.1016/S0305-0548(97)00031-2}.

\end{document}
