% Options for packages loaded elsewhere
\PassOptionsToPackage{unicode}{hyperref}
\PassOptionsToPackage{hyphens}{url}
%
\documentclass[
]{article}
\usepackage{lmodern}
\usepackage{amssymb,amsmath}
\usepackage{ifxetex,ifluatex}
\ifnum 0\ifxetex 1\fi\ifluatex 1\fi=0 % if pdftex
  \usepackage[T1]{fontenc}
  \usepackage[utf8]{inputenc}
  \usepackage{textcomp} % provide euro and other symbols
\else % if luatex or xetex
  \usepackage{unicode-math}
  \defaultfontfeatures{Scale=MatchLowercase}
  \defaultfontfeatures[\rmfamily]{Ligatures=TeX,Scale=1}
\fi
% Use upquote if available, for straight quotes in verbatim environments
\IfFileExists{upquote.sty}{\usepackage{upquote}}{}
\IfFileExists{microtype.sty}{% use microtype if available
  \usepackage[]{microtype}
  \UseMicrotypeSet[protrusion]{basicmath} % disable protrusion for tt fonts
}{}
\makeatletter
\@ifundefined{KOMAClassName}{% if non-KOMA class
  \IfFileExists{parskip.sty}{%
    \usepackage{parskip}
  }{% else
    \setlength{\parindent}{0pt}
    \setlength{\parskip}{6pt plus 2pt minus 1pt}}
}{% if KOMA class
  \KOMAoptions{parskip=half}}
\makeatother
\usepackage{xcolor}
\IfFileExists{xurl.sty}{\usepackage{xurl}}{} % add URL line breaks if available
\IfFileExists{bookmark.sty}{\usepackage{bookmark}}{\usepackage{hyperref}}
\hypersetup{
  pdftitle={Untitled},
  pdfauthor={Pablo Hidalgo García},
  hidelinks,
  pdfcreator={LaTeX via pandoc}}
\urlstyle{same} % disable monospaced font for URLs
\usepackage[margin=1in]{geometry}
\usepackage{graphicx,grffile}
\makeatletter
\def\maxwidth{\ifdim\Gin@nat@width>\linewidth\linewidth\else\Gin@nat@width\fi}
\def\maxheight{\ifdim\Gin@nat@height>\textheight\textheight\else\Gin@nat@height\fi}
\makeatother
% Scale images if necessary, so that they will not overflow the page
% margins by default, and it is still possible to overwrite the defaults
% using explicit options in \includegraphics[width, height, ...]{}
\setkeys{Gin}{width=\maxwidth,height=\maxheight,keepaspectratio}
% Set default figure placement to htbp
\makeatletter
\def\fps@figure{htbp}
\makeatother
\setlength{\emergencystretch}{3em} % prevent overfull lines
\providecommand{\tightlist}{%
  \setlength{\itemsep}{0pt}\setlength{\parskip}{0pt}}
\setcounter{secnumdepth}{-\maxdimen} % remove section numbering

\title{Untitled}
\author{Pablo Hidalgo García}
\date{29/7/2020}

\begin{document}
\maketitle

\hypertarget{introducciuxf3n}{%
\section{Introducción}\label{introducciuxf3n}}

En la compleja sociedad moderna aparecen retos que es necesario abordar
y solucionar de la mejor forma posible para que la vida sea fácil y
llevadera. Uno de esos muchos retos es la gestión de los residuos. La
sociedad de consumo moderna implica una generación de residuos que hay
que recoger y tratar para mejorar la vida en las ciudades y el impacto
mediambiental. En el año 2017 se recogieron en España más de 22.000
toneladas de residuos (alrededor de 460 kilogramos por habitante). Se
pueden distinguir dos formas en la que la recogida de los residuos se
puede mejorar. La primera de ellas es una inversión en aumentar los
recursos disponibles (por ejemplo, un aumento de la flota de camiones de
recogida); la segunda es optimizar la recogida con los recursos ya
existentes. Es en este segundo enfoque en el que se incide en este
trabajo, en particular en optimizar las rutas que recorren los camiones
con el objetivo de recoger la máxima cantidad de residuo conforme a las
restricciones de recursos.

El objetivo de este trabajo es el de aplicar algoritmos metaheurísticos
en la gestión de las rutas de los residuos como alternativa a lo
estudiado en (Expósito).

En este trabajo se ha utilizado un algoritmo híbrido entre la búsqueda
tabú y la búsqueda por vecindarios variables (Variable Neighborhood
search o VNS).

En el problema se han considerado al recogida independiente de dos tipos
de residuos: residuos de papel y cartón y residuos plásticos
identificados, habitualmente, con los colores marrones y amarillos,
respectivamente. Para cada tipo de residuo se considera un camión con un
punto de origen y un punto de destino (el almacén y la planta de
tratamiento de residuos) y una serie de puntos de recogida candidatos.
Cada punto de recogida tiene su propia tasa de llenado que determina el
nivel de residuo hasta el punto en el que esté saturado. La restricción
principal que hay que abordar es la del tiempo máximo de trabajo que los
operarios considerando un tiempo de 6.5 horas.

Durante el año 2017 se recogieron en España más de 22.000 toneladas de
residuos. La eficiencia en su recogida hace que

En el conjunto de la sociedad se presentan retos que es necesario
abordar. Por ejemplo, la gestión eficiente del tráfico o la gestión de
los residuos.

Vivimos en una sociedad cada vez más preocupada por el impacto

\hypertarget{vecindarios}{%
\section{Vecindarios}\label{vecindarios}}

\hypertarget{vecindario-1-auxf1adir}{%
\subsection{Vecindario 1: añadir}\label{vecindario-1-auxf1adir}}

Sea \(h\in\mathcal{H}\) un día del horizonte y \(r_h\) la ruta de ese
día tal que \(r_i=(o(\nu), p_1, p_2, \ldots, p_n, t(\nu))\).
Representamos todas las rutas como \(R=\{r_1, r_2,\ldots, r_{|H|}\}\).

Este vecindario busca añadir un nuevo punto \(p\in P(\nu)\) tal que el
incremento en la función objetivo sea máximo.

\hypertarget{vecindario-2-intercambiar-entre-duxedas}{%
\subsection{Vecindario 2: intercambiar entre
días}\label{vecindario-2-intercambiar-entre-duxedas}}

\hypertarget{vecindario-3-cambiar-puntos-de-recogida}{%
\subsection{Vecindario 3: cambiar puntos de
recogida}\label{vecindario-3-cambiar-puntos-de-recogida}}

\end{document}
