\PassOptionsToPackage{unicode=true}{hyperref} % options for packages loaded elsewhere
\PassOptionsToPackage{hyphens}{url}
%
\documentclass[]{article}
\usepackage{lmodern}
\usepackage{amssymb,amsmath}
\usepackage{ifxetex,ifluatex}
\usepackage{fixltx2e} % provides \textsubscript
\ifnum 0\ifxetex 1\fi\ifluatex 1\fi=0 % if pdftex
  \usepackage[T1]{fontenc}
  \usepackage[utf8]{inputenc}
  \usepackage{textcomp} % provides euro and other symbols
\else % if luatex or xelatex
  \usepackage{unicode-math}
  \defaultfontfeatures{Ligatures=TeX,Scale=MatchLowercase}
\fi
% use upquote if available, for straight quotes in verbatim environments
\IfFileExists{upquote.sty}{\usepackage{upquote}}{}
% use microtype if available
\IfFileExists{microtype.sty}{%
\usepackage[]{microtype}
\UseMicrotypeSet[protrusion]{basicmath} % disable protrusion for tt fonts
}{}
\IfFileExists{parskip.sty}{%
\usepackage{parskip}
}{% else
\setlength{\parindent}{0pt}
\setlength{\parskip}{6pt plus 2pt minus 1pt}
}
\usepackage{hyperref}
\hypersetup{
            pdftitle={Dudas del paper},
            pdfborder={0 0 0},
            breaklinks=true}
\urlstyle{same}  % don't use monospace font for urls
\usepackage[margin=1in]{geometry}
\usepackage{graphicx,grffile}
\makeatletter
\def\maxwidth{\ifdim\Gin@nat@width>\linewidth\linewidth\else\Gin@nat@width\fi}
\def\maxheight{\ifdim\Gin@nat@height>\textheight\textheight\else\Gin@nat@height\fi}
\makeatother
% Scale images if necessary, so that they will not overflow the page
% margins by default, and it is still possible to overwrite the defaults
% using explicit options in \includegraphics[width, height, ...]{}
\setkeys{Gin}{width=\maxwidth,height=\maxheight,keepaspectratio}
\setlength{\emergencystretch}{3em}  % prevent overfull lines
\providecommand{\tightlist}{%
  \setlength{\itemsep}{0pt}\setlength{\parskip}{0pt}}
\setcounter{secnumdepth}{0}
% Redefines (sub)paragraphs to behave more like sections
\ifx\paragraph\undefined\else
\let\oldparagraph\paragraph
\renewcommand{\paragraph}[1]{\oldparagraph{#1}\mbox{}}
\fi
\ifx\subparagraph\undefined\else
\let\oldsubparagraph\subparagraph
\renewcommand{\subparagraph}[1]{\oldsubparagraph{#1}\mbox{}}
\fi

% set default figure placement to htbp
\makeatletter
\def\fps@figure{htbp}
\makeatother


\title{Dudas del paper}
\author{}
\date{\vspace{-2.5em}}

\begin{document}
\maketitle

La ecuación \((2)\) espresa el nivel de llenado de un contenedor
\(c\in C\) transcurridos \(d\) días desde la última recogida:

\[
f(c, d) = \max(1, \quad q(c)\cdot d + b(c))
\]

Entiendo que la función es \(f:C\times D\rightarrow [0,1]\). Mi duda es
si en vez del máximo no debería ser el mínimo en la función anterior.
Además, en el párrafo siguiente se especifica que
\(q(c)\cdot d + b(c) > 1\) supone el desborde del contenedor.

La ecuación \((10)\) aparece como

\[
um_{v\in \mathcal{V}} Y_{ih}^v \leq 1, \quad i\in P, h\in\mathcal{H}
\]

Imagino que será un error de la transcripción en Latex y que sería un
sumatorio, ¿verdad?

\[
\sum_{v\in \mathcal{V}} Y_{ih}^v \leq 1, \quad i\in P, h\in\mathcal{H}
\]

Entiendo qué trata de representar el conjunto \(\mathcal{D}_{ih}\) pero
no termino de entender cómo interpretar la fórmula \((14)\):

\[
\mathcal{D}_{ih} = \left\{ d < \frac{h}{\sum_{v\in \mathcal{V}} Y_{id}^v} \right\}
\]

En la ecuación \((15)\), \(\mathcal{H}_i\) se refiere al número de días
que lleva el contenedor \(i\in C\) sin recogerse al inicio del horizonte
temporal. En el párrafo justo depués de la ecuación \((4)\) ¿no se
define este valor como \(d_i\)?

La función objetivo definida en \((16)\), ¿no debería ser una
maximización?

\[
\max \sum_{v\in \mathcal{V}} \sum_{i\in P}\sum_{h\in \mathcal{H}} f_i(\eta_{ih})\cdot Y_{ih}^v
\]

Se asume que el llenado de los contenedores se produce de forma lineal.
Esta asunción se hace basada en datos históricos o se hace alguna
estimación (por ejemplo, regresión lineal)?¿Cómo se calcula la tasa de
llenado \(q(c)\)?

En la obtención de las distancias obtenidas de la API Distance Matrix de
Google Maps, ¿se tiene en cuenta el día o es un valor genérico?

EN la comparativa entre el modelo matemático y la solución heurística
(tabla 1), el día el modelo matemático tiene un nivel de llenado
inferior. Supongo que, teóricamente, en este día, el modelo matemático
debería comportarse mejor y, si no es así, es por el límite de 2 horas
en la ejecución del CEPLEX.

\end{document}
